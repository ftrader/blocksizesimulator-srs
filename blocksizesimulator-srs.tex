%Software Requirements Specification for Adjustable Blocksize Limit Scenario Simulator

%Based on SRS-Tex template:
%Copyright 2014 Jean-Philippe Eisenbarth
%This program is free software: you can
%redistribute it and/or modify it under the terms of the GNU General Public
%License as published by the Free Software Foundation, either version 3 of the
%License, or (at your option) any later version.
%This program is distributed in the hope that it will be useful,but WITHOUT ANY
%WARRANTY; without even the implied warranty of MERCHANTABILITY or FITNESS FOR A
%PARTICULAR PURPOSE. See the GNU General Public License for more details.
%You should have received a copy of the GNU General Public License along with
%this program.  If not, see <http://www.gnu.org/licenses/>.

%Based on the code of Yiannis Lazarides
%http://tex.stackexchange.com/questions/42602/software-requirements-specification-with-latex
%http://tex.stackexchange.com/users/963/yiannis-lazarides
%Also based on the template of Karl E. Wiegers
%http://www.se.rit.edu/~emad/teaching/slides/srs_template_sep14.pdf
%http://karlwiegers.com
\documentclass{scrreprt}
\usepackage{listings}
\usepackage{underscore}
\usepackage[bookmarks=true]{hyperref}
\usepackage[utf8]{inputenc}
\usepackage[english]{babel}
\hypersetup{
    bookmarks=false,    % show bookmarks bar?
    pdftitle={Software Requirement Specification},    % title
    pdfauthor={Jean-Philippe Eisenbarth},                     % author
    pdfsubject={TeX and LaTeX},                        % subject of the document
    pdfkeywords={TeX, LaTeX, graphics, images}, % list of keywords
    colorlinks=true,       % false: boxed links; true: colored links
    linkcolor=blue,       % color of internal links
    citecolor=black,       % color of links to bibliography
    filecolor=black,        % color of file links
    urlcolor=purple,        % color of external links
    linktoc=page            % only page is linked
}%
\def\myversion{0.1 }
\date{}
%\title
\usepackage{hyperref}
\begin{document}

\begin{flushright}
    \rule{16cm}{5pt}\vskip1cm
    \begin{bfseries}
        \Huge{SOFTWARE REQUIREMENTS\\ SPECIFICATION}\\
        \vspace{1.9cm}
        for\\
        \vspace{1.9cm}
        Adjustable Blocksize Limit Scenario Simulator\\
        \vspace{1.9cm}
        \LARGE{Version \myversion (draft)}\\
        \vspace{1.9cm}
        Prepared by Bitcoin Cash developers\\
        \vspace{1.9cm}
        \today\\
    \end{bfseries}
\end{flushright}

\tableofcontents


\chapter*{Revision History}

\begin{center}
    \begin{tabular}{|c|c|c|c|}
        \hline
	    Name & Date & Reason For Changes & Version\\
        \hline
	    imaginary_username & 2019-05-12 & initial text requirements & n/a\\
        \hline
	    freetrader & 2019-05-22 & initial SRS draft & 0.1\\
        \hline
    \end{tabular}
\end{center}

\chapter{Introduction}


\section{Purpose}
This document describes all the requirements for a blocksize simulator
(hereafter shortened to `simulator').

These requirements do not yet map to an actual implementation release
as the simulator has not yet been developed.

This SRS describes the entire simulator system.


\section{Document Conventions}
Certain incomplete passages or sections are marked as `To be completed (TBC)',
`To be determined (TBD)' or with a TODO notice.


\section{Intended Audience and Reading Suggestions}
This document is intended for software developers and users of the simulator.

Software developers are encouraged to develop their own implementations based
on these requirements.

Users may find the requirements useful to understand what the simulator is
intended to do.



\section{Project Scope}
The purpose of the simulator is to test a narrow class of adjustable
blocksize schemes for Bitcoin Cash, where consensus blocksize limits are
adjusted with only constants and actual blocksizes as inputs.

Using a simplified model of a Bitcoin Cash network comprised of resource-constrained
but upgradeable mining and service nodes, and including sources of `legitimate'
and `spam' transactions as well as `poison block' providers, the simulator is
intended to provide information on the following observables:

\begin{itemize}
   \item response characteristics of various dynamic blocksize algorithms
   \item induced transaction delays in face of full blocks (whether from legit or spam traffic)
   \item memory pool (mempool) desynchronization
   \item network splits
   \item node downtime (both service and mining)
   \item upgrade feasibility
   \item cost needed for a malicious actor to incur damage to the network
\end{itemize}


\section{References}
\begin{itemize}
   \item text file with rough requirements ompiled by imaginary_username
   \item private Slack discussion notes with clarification of rough requirements
   \item Asymmetric Moving Maxblocksize Based On Median

      (https://github.com/bitcoincashorg/bitcoincash.org/pull/149)
\end{itemize}

\chapter{Overall Description}


\section{Product Perspective}
The simulator is intended to inform the selection of a dynamic blocksize algorithm
for Bitcoin Cash.

It is intended to be standalone, self-contained, easily adaptable and
re-usable in future.

Test scenarios and algorithms subjected to simulation may be maintained
outside of the simulator project.


\section{Product Functions}

Basic description of the main functions:
\begin{itemize}
   \item Interpret command line arguments provided by user to start a simulation run
   \item Read scenario inputs from user-provided text files
   \item Optionally, generate a block history as initial condition for the simulation
   \item Run the network simulation for n blocks, optionally outputting traces to console or log file
   \item Stop and output simulation results (to console or disk)
\end{itemize}



\section{User Classes and Characteristics}
The simulator is intended to be used by anyone with an interest in Bitcoin Cash
dynamic blocksize algorithms.

This includes protocol developers, miners, service providers and hobbyists.

Operation of the simulator will require only editing of text files and issuance of
commands in a shell (i.e. at least initially it will be an interpreted command line
program).

Interpretation of the simulator output will likely require domain expert knowledge.


\section{Operating Environment}
The simulator is intended to run on reasonably fast commodity personal
computer hardware equipped a suitable interpreter and any required libraries.


\section{Design and Implementation Constraints}
To be decided (TBD)


\section{User Documentation}
No separate user doc planned aside from this specification and some commented
example scenarios.

Initial use of the simulator should yield more scenarios which can serve as
reference for future use.


\section{Assumptions and Dependencies}
It is assumed that the adaptive algorithms to be evaluated can be developed
by others according to a well-specified interface definition.

At least initially, certain simplifying assumption can be made in terms
of the transaction minimum fee rate (e.g. assume 1 sat/byte flat rate).

The node upgrading aspects of the simulation can also be stubbed initially.


\chapter{External Interface Requirements}

\section{User Interfaces}
Initially, the simulator user interface will be text console driven with input
provided by command line and configuration text files, and output written to
console and/or disk.

\section{Hardware Interfaces}
None.

\section{Software Interfaces}
The software is stand-alone and uses OS facilities for console and file I/O.

Except for an interface to the adaptive blocksize algorithms to be evaluated,
no interfaces to other software are planned.

\subsection{Scenario Files}
Precise structure: TBD - might be script files in the interpreter language
at first which can simply be pulled in and evaluated.

A more structured scenario data format may be implemented after the core
simulation aspects have been established.


\subsection{Blocksize algorithms}

To be specified as an interface which has to be implemented for a node.
It will have access to the node's view of the blockchain in order to calculate
the maximum block size (`hard cap') for the next block based on past block sizes.

The same algorithm will be used by all the nodes in a simulation run.

The algorithms to be evaluated are not considered part of this simulator
specification.

Since the simulator will need to come with some working default blocksize
algorithm, the standard fixed blocksize limit of 32MB in the current
Bitcoin Cash protocol can be implemented as an example instance.


\section{Communications Interfaces}
None.


\chapter{System Features}

This section describes the main features of the simulation and lists the
functional requirements associated with them.

To shorten the document, the `Priority` subsections have been removed
from the individual feature sections. All matters of feature priority should
be described here.

All features are considered high priority except for the following exceptions:

\begin{itemize}
   \item Node upgrade process implementation (Medium priority)
   \item Generic nodes (Low priority)
\end{itemize}


\section{Feature: General}

\subsection{Description}

This section describes general requirements which are not related to
specific features.


\subsection{Stimulus/Response Sequences}

The simulation runs must be deterministic to allow reproduction.

As some random processes are modeled, a random number generator must be
initialized with a seed that is made known to the user.

A specific seed can also be provided by the user as part of the scenario
(optionally overriden via command line to simplify repeat trials of the
a particular scenario under randomness).


\subsection{Functional Requirements}


\subsubsection{BLOCKSIM-GENERAL-REQ-1 (RNG seed)}

The simulation shall initialize a random number generator (RNG) with a seed
provided by the user in the scenario or on the command line.

If no seed is provided, a seed will be generated from some local machine
parameters and the epoch time, and the seed will be printed to allow
a re-run by the user.

\subsubsection{BLOCKSIM-GENERAL-REQ-2 (termination condition)}

The simulation shall terminate once a block of specified height has been
evaluated by each node.

Example: stop once all nodes have processed 10,000 blocks.

Once the simulator stops, it outputs some statistics (see Output feature).

This setting shall have no default - the simulator shall abort prior
to starting and complain if no block height at which to stop has been
specified in the scenario or on the command line.

\subsubsection{BLOCKSIM-GENERAL-REQ-3 (accounting unit)}

Certain entities in the simulation need to account economically for
their actions.

This applies at minimum to poison block spammers, mining and service nodes.

For these calculations, the simulation shall use a fictional unit of account:
the Sim Dollar, amounts of which are denoted in output by a preceding dollar sign.


\section{Feature: Network}

\subsection{Description}

The simulation requires modeling of a network layer on which
transactions and blocks are transported.


\subsection{Stimulus/Response Sequences}

Traffic generators create transactions which are relayed to all nodes.

Mining nodes create blocks which are relayed to all other nodes (mining, service and generic).

Propagation time of transactions and blocks is not modeled (at least, initially).

That means blocks and transactions take only the minimum amount of time (1 second) to
reach other entities in the simulation.


\subsection{Functional Requirements}


\subsubsection{BLOCKSIM-NETWORK-REQ-1 (instantaneous transaction dissemination)}

Any transactions added to the network by traffic generators at a specific time step shall be inserted into the receive
queue of all nodes instantaneously (at the current simulation time step).

\subsubsection{BLOCKSIM-NETWORK-REQ-2 (instantaneous block dissemination)}

Any block added to the network by a mining node at a specific time step shall be inserted into the receive
queue of all other nodes instantaneously (at the current simulation time step).



\section{Feature: Time}

\subsection{Description}

The simulation requires modeling of time (for example w.r.t. the
processing of blocks by nodes).

Certain attributes of simulation entities such as rates of transactions
created by traffic generators, or processing units required by a node
to validate a transaction or block, may be specified most naturally in terms
of some units per time.

Traffic generators will generate transactions to be inserted at a given time
in the simulation (which depends on each generator's characteristics).

Miners will stochastically generate blocks (provided they have transactions
in their mempool). Once a miner finds a block, it adds that block onto
the network immediately.


\subsection{Stimulus/Response Sequences}

At each timestep the simulator shall evaluate the production of new
transactions and blocks, the upgrade decisions of eligible nodes,
the processing of input data by active nodes and state transitions
of nodes which may drop out of the network.


\subsection{Functional Requirements}


\subsubsection{BLOCKSIM-TIME-REQ-1 (simulation time unit)}

The smallest time unit in the simulation will be one (1) second.

\subsubsection{BLOCKSIM-TIME-REQ-2 (meaning of negative/positive and zero sim times)}

The starting time of the simulation will be referred to as time zero
(t=0), all future events will have an integer simulation time $>$ 0
assigned to them.

Events described in scenarios must be expressed in terms of this
simulation time.

Events which occur before the simulation start (e.g. pre-existing blocks
which as part of initial conditions) shall have negative integer simulation
times assigned to them (e.g. t=-1, t=-43, etc.)



\section{Feature: Scenarios}

\subsection{Description}

A scenario completely defines all the inputs for a simulation run.

It may be specified in one single file or a collection of files (TBD),
but in either case there shall be one single file that acts as the
main starting point to provide the input data used in a simulator run.

Some example scenario files can be provided with the base simulator.

Editing scenario files is outside the scope of this SRS. As they will be pure
text files, it can be left to the user and their favorite text editor.


\subsection{Stimulus/Response Sequences}

When users start the simulator, they will need to specify the (main)
scenario file on the command line.

The simulator processes the scenario file and any potential dependencies.

If there are no errors (such as missing or invalid data), the simulation
will run, otherwise the simulator should exit with appropriate error messages.


\subsection{Functional Requirements}


\subsubsection{BLOCKSIM-SCENARIO-REQ-1 (starting scenario file)}

The simulator shall read in the scenario file provided on the command line and
automatically process any dependencies (further included scenario files).


\subsubsection{BLOCKSIM-SCENARIO-REQ-2 (abort if invalid/incomplete scenario handling)}

The simulator shall abort if any required input information is missing or
invalid after processing the scenario file(s).

\subsubsection{BLOCKSIM-SCENARIO-REQ-3 (inform which data is invalid/incomplete)}
Any missing/invalid scenario information shall be listed to the user to enable
them to adequately configure the scenario.


\subsubsection{BLOCKSIM-SCENARIO-REQ-4 (list of scenario data items)}
Input data items specified by scenario shall be as follows
(the list below should be complete):

TODO: some may be optional - need to describe better

\begin{enumerate}
   \item Seed for RNG initialization. This may be overridden by a command line parameter.
   \item block height at which to terminate the simulation. This may be overridden by a command line parameter.
   \item Blocksize algorithm to use in nodes (this also defines the `hard' max blocksize for a node)
   \item List of pre-historical block sizes (for blocks preceding the first mined block)
   \item Definition of the node population
      \begin{enumerate}
         \item Number of mining nodes
         \item Number of service nodes
         \item Number of generic nodes (optional - generic nodes would be non-mining, non-service nodes)
         \item Default node mempool size (in bytes)
         \item Node interuption tolerance threshold (in seconds). If the period where a node drops out exceeds a tolerance threshold, the node is removed from the network permanently.
      \end{enumerate}
   \item Definition of the mining nodes
   \item Definition of the service nodes
   \item Definition of the traffic generators (these are not nodes, but sources which inject transactions traffic into the simulation)
      \begin{enumerate}
         \item sources of `legit' transactions
         \item sources of `spam' transactions
      \end{enumerate}
\end{enumerate}



\section{Feature: Node class}

\subsection{Description}

The node class provides the base from which all node types in the
system (i.e. mining, service, generic nodes) are derived.


\subsection{Stimulus/Response Sequences}

After the scenario has been read and validated, the simulator will
instantiate node objects of the specified types before starting the
simulation.

Specialized node types may override some of the resource attributes
of the node base class.

Node objects maintain their own mempool and view of the blockchain history,
and receive transactions (from traffic generators) and blocks (from miners)
for processing.

If their resources or limits are exceeded during simulated operation, nodes
can become unavailable (`downtime`) and may require simulated maintenance (TBD).

The transactions in received blocks must be entered into the node's mempool
before the block can be processed.

Blocks that have been stuck on a Poison block for too long must be able to
fall back to processing another legitimate block instead.

Validation time of received transactions (as opposed to blocks) before
entering a node's mempool is not modeled explicitly.

\subsection{Functional Requirements}

\subsubsection{BLOCKSIM-NODE-BASE-REQ-1 (node type identifier)}

Each nodes shall be identified by an attribute as one of the following types:

\begin{itemize}
   \item regular mining node
   \item poison mining node
   \item service node
   \item generic node
\end{itemize}


\subsubsection{BLOCKSIM-NODE-BASE-REQ-2 (mempool per node)}

Nodes shall maintain their own mempool containing transactions already seen
but not removed by a block in the node's active blockchain history.


\subsubsection{BLOCKSIM-NODE-BASE-REQ-3 (blockchain history per node)}

Nodes shall maintain their own history of the blockchain, as a list of
linked blocks which the node has been able to `validate`.


\subsubsection{BLOCKSIM-NODE-BASE-REQ-4 (validation rate per node)}

Nodes shall have an overrideable attribute which defines a processing rate
at which they are able to validate blocks.

This rate may be defined in terms of some unit (e.g. some `cycles` or `ops`)
used by blocks to specify their validation cost.


\subsubsection{BLOCKSIM-NODE-BASE-REQ-5 (maximum processable regular block size)}

Nodes shall have an overrideable attribute which defines a maximum processable
block size (in bytes) for regular blocks.


\subsubsection{BLOCKSIM-NODE-BASE-REQ-6 (maximum processable poison block size)}

Nodes shall have an overrideable attribute which defines a maximum processable
block size (in bytes) for poison blocks. Its value must be greater than or
equal the maximum size of regular blocks.


\subsubsection{BLOCKSIM-NODE-BASE-REQ-7 (input queue for blocks)}

Nodes shall maintain a queue for blocks received from miners but not
yet validated.

NOTE: This is necessary because nodes have to process the blockchain
sequentially, but may be receiving blocks of later from miners while they
are still processing earlier blocks.


\subsubsection{BLOCKSIM-NODE-BASE-REQ-8 (parallel validation of same-height blocks)}

Nodes shall validate received blocks in parallel according to their
configured validation rates and the validation cost of each block.

Such validation is subject to mempool constraints.

Once a node has validated a block at a given height, it aborts validation
of other candidate blocks for the same height.


\subsubsection{BLOCKSIM-NODE-BASE-REQ-9 (mempool space required for block validation)}

Validation is subject to mempool constraints.
Transactions not already received into the mempool (e.g. those in a `poison block`)
must be entered into node mempool before the block can be validated.

NOTE: as per general node requirements, if the mempool capacity is reached
the node is considered to have gone down (into maintenance).


\subsubsection{BLOCKSIM-NODE-BASE-REQ-10 (parameter to dropout when time exceeded)}

A node's definition shall include a boolean configuration parameter which
controls whether it drops into maintenance if a block's validation time exceeds 10 minutes.


\subsubsection{BLOCKSIM-NODE-BASE-REQ-11 (action if block validation $>$ 10 minutes)}

If a node is unable to process a block within a simulation time of 10 minutes,
it shall, depending on the aforementioned  boolean configuration parameter,
either abort the processing of the block and carry on processing alternative
candidates, or drop out.

NOTE: This allows modeling a node `giving up' on a `poison block` after a
cut-off period, or go down entirely if the node gets stuck forever or crashes
on such a block.


\subsubsection{BLOCKSIM-NODE-BASE-REQ-12 (node mempool cap must $>$ node max blocksize)}

A node's configured mempool capacity shall exceed the configured
maximum blocksize that can be processed by the node, otherwise a fatal
misconfiguration error shall be reported and the simulation aborted.


\subsubsection{BLOCKSIM-NODE-BASE-REQ-13 (drop low-fee txs if full mempool)}

As long as a node's mempool is full, incoming transactions (not in blocks) shall
evict lower-fee transactions from the node's mempool.

NOTE: this may result in mempool desynchronization

\subsubsection{BLOCKSIM-NODE-BASE-REQ-14 (recompute cap after extending chain)}

Whenever a node has finished extending its active chain, it shall invoke
the scenario-specified adaptive blocksize algorithm to obtain a newly
computed maximum block size (`hard cap').


\section{Feature: Generic nodes}

\subsection{Description}

Generic nodes are neither mining nor economic nodes, and are modeled
after relay nodes which have no specific purpose.

Simulating them may not be particularly useful, but might give some
insight at what level of network utilization the low-spec nodes will
drop out.


\subsection{Stimulus/Response Sequences}

Generic nodes process incoming transactions and blocks as they are
produced.

They do not need revenue / cost of maintenance accounting.

The statistics tracked are inherited from the base node class.

Also, the basic node rules for processing blocks and for dropping out due to
exceeded mempool capacity apply.


\subsection{Functional Requirements}

\subsubsection{BLOCKSIM-NODE-GENERIC-REQ-1 (no cost/revenue for generic nodes)}

A generic node shall not be required to do any revenue / cost of maintenance accounting.



\section{Feature: Mining nodes}

\subsection{Description}

Mining nodes are derived from generic nodes. They inherit all the
attributes and behaviors except as defined by the additional functional
requirements below.

The main additional capability is to produce blocks.

Two different kinds of mining nodes are modeled:

\begin{enumerate}
   \item Regular miners: they produce blocks based on their mempool which is filled from `external' traffic generators. They represent the `honest miners' that are not trying to harm the system.
   \item `Poison block' miners: they produce abnormal blocks filled with transactions that have not been propagated over the network in the regular manner. Their poisonous blocks are also much more costly to validate - the aim of these blocks is to harm the rest of the network. An attack by a poison block miner is not expected to last more than a day. A poison block miner always prefers to build on a fork chain containing his own poison blocks.
\end{enumerate}

Regular mining nodes generate revenue by way of the fees from transactions
in a mined block. Block rewards are ignored in the model.


\subsection{Stimulus/Response Sequences}

As a simplification, hashpower and difficulty are not modeled, therefore
all mining nodes that are not `down' will have an equal chance of finding
the next block (hashpower of all nodes is assumed equal, and difficulty
is assumed such that the next block will be found on average within 10
minutes).

Regular blocks are generated every 10 minutes, uniformly, by a randomly chosen
regular miner.

If there is a poison block miner, it injects its poison blocks into the network
at five-minute marks in between the regular blocks.


\subsection{Functional Requirements}


\subsubsection{BLOCKSIM-NODE-MINING-REQ-1 (regular vs poison miner)}

A mining node definition shall include a mandatory parameter to denote
it as either a regular miner or a poison block miner.

\subsubsection{BLOCKSIM-NODE-MINING-REQ-2 (only one poison block miner)}

At most one miner in a scenario shall be designated as a poison block
miner.

\subsubsection{BLOCKSIM-NODE-MINING-REQ-3 (regular blocks produced every 10 minutes)}

As a simplifying assumption, the regular miners shall mine one block uniformly
every 10 minutes.

i.e. the first regular block will be produced at t=600.

\subsubsection{BLOCKSIM-NODE-MINING-REQ-4 (random choice of regular miner producing block)}

The regular miner that produces the next block shall be chosen at random
from the population of regular miners that are active at the time the block
needs to be created.


\subsubsection{BLOCKSIM-NODE-MINING-REQ-5 (regular mining node soft cap)}

Regular mining nodes will have a `soft cap' attribute configured by default to
1/4 of the node's hard cap, and which will limit the size of blocks produced
by the mining node.


\subsubsection{BLOCKSIM-NODE-MINING-REQ-6 (meaning of zero soft cap)}

A value of zero (0) for `soft cap' parameter shall mean that a regular mining
node can produce blocks as big as its its hard limit.


\subsubsection{BLOCKSIM-NODE-MINING-REQ-7 (fill to soft cap unless zero)}

A regular mining node will attempt to fit in as many transactions from its mempool
as possible without exceeding its soft cap if that is zero.

NOTE: a soft cap of 0 means fill to the hard cap.


\subsubsection{BLOCKSIM-NODE-MINING-REQ-8 (traffic agnostic)}

A regular mining node shall not discriminate between `legit' and `spam' transactions
when filling a block.

\subsubsection{BLOCKSIM-NODE-MINING-REQ-9 (minfee for mining)}

Transaction which do not carry a minimum fee rates as expected by a regular mining
node shall not mined by that mining node.


\subsubsection{BLOCKSIM-NODE-MINING-REQ-10 (drop arriving below minfee txs)}

Transactions that arrive outside of a mined block and carrying a fee below
the minfee of a mining node shall not be entered into its mempool.

For the purpose of the simulation they can be considered `dropped'.

\subsubsection{BLOCKSIM-NODE-MINING-REQ-11 (poison block mining parameters)}

A poison block mining node definition shall include the following mandatory
parameters describing the poisonous blocks it produces:

\begin{itemize}
   \item positive integer: time interval offset w.r.t to next regular blocks, in seconds. Defaults to 300s. (5 minutes)
   \item positive integer: height of the regular block in the simulation after which production of poison blocks will commence. If 0 the first poison block may be issued before the first regular block. If $>$ 0 then it will be issued after the respective regular block, with an offset given by the previous parameter.
   \item positive integer in range 1..144: number of poison blocks to issue (i.e. not more than a day's worth)
   \item positive integer: validation cost per transaction in the poisonous block (this can be higher than the standard cost to validate transactions produced by traffic generators)
   \item positive integer: size of poison block (in bytes). The poison block is filled with as many transactions as fit into this size. (TODO: specify standard size of a transaction)
\end{itemize}

\subsubsection{BLOCKSIM-NODE-MINING-REQ-12 (regular mining revenue)}

When a regular mining node mines a block, the fees of all transactions
(legit and spam) in the block shall be counted as revenue for the node.

\subsubsection{BLOCKSIM-NODE-MINING-REQ-13 (regular mining cost)}

Cost of operating a regular mining node shall be based on the validation
cost of blocks and transactions.

TBD: precise formula

\subsubsection{BLOCKSIM-NODE-MINING-REQ-14 (poison mining revenue)}

Poison mining nodes shall not generate any revenue.

\subsubsection{BLOCKSIM-NODE-MINING-REQ-15 (poison mining cost)}

Cost of operating a poison mining node shall be based on the validation
cost of incoming blocks and transactions.

TBD: precise formula (same as for BLOCKSIM-NODE-MINING-REQ-13)


\section{Feature: Service nodes}

\subsection{Description}

Service nodes are used to model economic service providers, and thus
are simulated with a focus on measuring cost of downtime and feasibility
of upgrades, similar to mining nodes, despite service nodes not generating
any blocks.

Service nodes generate revenue which is accounted, modeled as a function of
`legit` transactions they successfully process.

NOTE: The revenue/cost calculation formulas are elaborated in the functional
requirements of this feature section.

This includes both validated transactions accepted into mempool as well as
the transactions in blocks.

Service nodes also accrue maintenance costs, modeled as a function of certain
capacities of the node over time.


\subsection{Stimulus/Response Sequences}

Upgrade decisions are taken before any other processing done by the node.

Then transactions queues are cleared towards the node's mempool

\subsection{Functional Requirements}

\subsubsection{BLOCKSIM-NODE-SERVICE-REQ-1 (drop arriving below minfee txs)}

Transactions that arrive outside of a mined block and that carry a fee below
the minfee of a service node shall not be entered into its mempool.

For the purpose of the simulation they can be considered `dropped'.


\subsubsection{BLOCKSIM-NODE-SERVICE-REQ-2 (service node revenue)}

A service node shall earn revenue based on the number of legitimate
transactions it (a) stores in its mempool  (b) clears via blocks validation.

TBD: precise formula

\subsubsection{BLOCKSIM-NODE-SERVICE-REQ-3 (service node cost)}

A service node shall accrue operating cost based on its principal
capacities ("the more powerful the node the costlier to run").

NOTE: The precise formula is TBD, but could look more or less like this:

maintenance cost/day = C1*mempool capacity + C2*maximum validation size + C3*validation speed per tx



\section{Feature: Transactions}

\subsection{Description}

Transactions are modeled in a simplified way, without inputs/outputs.

Transactions have associated a size (in bytes), a validation cost incurred
during validation of a block by a node, and a fee collected by the node that
mines it.

Two kinds of transactions will be simulated:

\begin{itemize}
   \item `legit` transactions
   \item `spam` transactions
\end{itemize}

Nodes will not distinguish between these, but they shall be distinctly
tagged to allow accounting of the impact of spamming on the system.


\subsection{Stimulus/Response Sequences}

Transactions are inserted onto the network by traffic generators which are
polled at each simulation step.

They enter instantly into transaction queues of all nodes except poison block
miners.

Relaying of transactions is not further modeled, nodes only care about
them in terms of their mempool capacities and fees to collect
(if a regular miner).


\subsection{Functional Requirements}

\subsubsection{BLOCKSIM-TRANSACTIONS-REQ-1 (tx size)}

A transaction shall have an associated positive size (in bytes).

\subsubsection{BLOCKSIM-TRANSACTIONS-REQ-2 (tx fee)}

A transaction shall have an associated positive fee payable to a miner.

\subsubsection{BLOCKSIM-TRANSACTIONS-REQ-3 (tx validation cost)}

A transaction shall have an associated positive validation cost (in a suitable unit e.g. cycles).



\section{Feature: Blocks}

\subsection{Description}

The simulation will be initialized with a certain amount of "history" of
pseudo-blocks of negative height, of which only the size is known.

The sizes of these historical blocks shall be configurable via scenario.

Newly mined blocks are linked into a chain via a block id and a reference
to their parent block id.

No hashing needs to be performed.

Regular blocks must include a list of transactions.

Poison blocks can be represented in a simplified way which allows huge
blocks to be represented using less space, since their individual
transactions are never propagated and can thus be abstracted.


\subsection{Stimulus/Response Sequences}

When a mining node generates a block, it links it to the tip of the chain
of blocks on which it is building.


\subsection{Functional Requirements}

\subsubsection{BLOCKSIM-BLOCKS-REQ-1 (common block structure)}

A block shall have a block id of its own and a `previous block` id.


\subsubsection{BLOCKSIM-BLOCKS-REQ-2 (regular block structure)}

A regular block shall have an associated set of transactions.


\subsubsection{BLOCKSIM-BLOCKS-REQ-3 (poison blocks structure)}

A poison block shall have the following attributes beyond the linking ids:

\begin{enumerate}
   \item number of transactions
   \item size in bytes per transaction
   \item fee per transaction
\end{enumerate}

This allows to compute the block size and its impact on local mempools
as well as accounting.

\subsubsection{BLOCKSIM-BLOCKS-REQ-4 (configurable block prehistory)}

A list of prehistorical blocks forming a chain preceding the first mined
block (denoted by height 0) shall be configurable via scenario file.

NOTE: these prehistorical blocks shall have negative block heights

\subsubsection{BLOCKSIM-BLOCKS-REQ-5 (only size for prehistorical blocks)}

The list of prehistorical blocks shall record only the size of each such
block.

NOTE: the size of the prehistorical block and its position in the chain
(which is inferred by its position in the prehistory list) is the only thing
that matters to the adaptive block size algorithms under evaluation.


\section{Feature: Traffic generators}

\subsection{Description}

The simulation must provide for the generation of transactions (legit and spam).

Traffic generators perform this task.


\subsection{Stimulus/Response Sequences}

Traffic generators are polled at each simulation step to generate more
transactions according to their configured parameters.

These transactions are entered into the mempools of all nodes that are not
down for maintenance.


\subsection{Functional Requirements}

\subsubsection{BLOCKSIM-TRAFFICGEN-REQ-1 (background level)}

A traffic generator shall have a configurable background transaction level
(specified in transactions per second).

This is the base rate at which the generator emits transactions.

\subsubsection{BLOCKSIM-TRAFFICGEN-REQ-2 (spike configuration)}

A traffic generator definition shall allow the following optional set of parameters to configure
traffic spikes:

\begin{enumerate}
   \item (initial) spike level: peak rate at which the generator emits transactions, specified as positive number of transactions per second (value must exceed generator's background level)
   \item spike ramp-up duration : non-negative integer time in seconds to scale from background to spike output level
   \item spike sustain duration: positive time in seconds to sustain the spike's peak output
   \item spike spike activation interval: trigger spike at these intervals (positive number of seconds $>$ twice the ramp-up + sustain duration. The end of the last spike is considered to be the start of a new interval.
   \item spike spike activation random offset: non-negative integer number of seconds. The actual spike is triggered a random number of seconds, up to this parameter value, prior or after to the regular scheduled interval trigger. If the value of the random offset is zero, it means no randomization around the interval happens, i.e. the spike is triggered at regular intervals. The value may not exceed the activation interval.
\end{enumerate}

If any of these parameters are configured, they must all be configured.
There are no default values.


\subsubsection{BLOCKSIM-TRAFFICGEN-REQ-3 (background and spike traffic rates of increase)}

A traffic generator shall have configuration parameters for the rates
of increase, over time, of its background and spike transaction levels.

These shall be non-negative real numbers value (with default values of 0)
expressing increase in transactions per second per second. The default
is no increase at all.

If specified, increases shall only apply to the spike peak level and not the
duration or any other time-related parameters.


\subsubsection{BLOCKSIM-TRAFFICGEN-REQ-4 (generated transaction type)}

A traffic generator shall have a configuration parameter to
identify the type of transactions it generates as either `legit' or `spam'.

All transactions generated by an individual traffic generator shall
be marked as belonging to its configured transaction type.


\subsubsection{BLOCKSIM-TRAFFICGEN-REQ-5 (configure fee levels)}

A traffic generator shall have a configuration parameter to
provide for a table of what percentage of transactions carry a certain fee.

The fee levels on a traffic generator remain static as defined in the scenario.
There is no requirement to modify them during the simulation.


\subsubsection{BLOCKSIM-TRAFFICGEN-REQ-6 (generate txs according to fee levels)}

A traffic generator shall generate transactions with fees distributed
according to the fee levels which have been configured for it.


\subsubsection{BLOCKSIM-TRAFFICGEN-REQ-7 (spam budget)}

A traffic generator which is configured to emit `spam' transactions
shall have a configuration parameter to specify its initial budget
(in Sim dollars) for transaction fees.


\subsubsection{BLOCKSIM-TRAFFICGEN-REQ-8 (spam nodes deplete budget)}

A traffic generator which is configured to emit `spam' transactions
shall subtract the cost of emitted transaction fees from its available
spam budget.

NOTE: non-spam traffic generators do not have to account for their fees.




\section{Feature: Node upgrades}

\subsection{Description}

At certain upgrade intervals, node "operators" will consider to upgrade
their nodes if the foresee a need and if it makes economic sense
(node revenue exceeds cost).


\subsection{Stimulus/Response Sequences}

If upgrades are enabled for a node, an upgrade interval timer determines
when the node takes the next upgrade decision.

The decision to upgrade is evaluated when simulation control passes to
consider the node, before any other actions (e.g. processing queued
transactions or blocks).


\subsection{Functional Requirements}

\subsubsection{BLOCKSIM-UPGRADES-REQ-1 (upgrade interval parameter per node)}

A node shall allow an associated upgrade interval (non-negative integer in seconds).
If the value of the parameter is 0 (zero), no upgrades shall be undertaken
for the node.

\subsubsection{BLOCKSIM-UPGRADES-REQ-2 (node class upgrade parameters)}

If no node-specific upgrade interval is specified, a node shall inherit
a class-wide node interval configuration parameter (i.e. a parameter
specifying the upgrade interval for an entire class of nodes (mining,
service, generic).

\subsubsection{BLOCKSIM-UPGRADES-REQ-3 (upgrade conditions)}

A node shall only upgrade if ALL of the following conditions are met:

\begin{enumerate}
   \item number of (legit + spam transactions) as received by the node has increased
over a "consideration period" (configurable duration in seconds)

   \item revenue $>$ cost

   \item after the upgrade, revenue would still exceed cost
\end{enumerate}


\subsubsection{BLOCKSIM-UPGRADES-REQ-3 (upgrade effect)}
If the upgrade conditions are all met, a node shall upgrade to a capacity
which is estimated to be enough until the next evaluation.

TODO: this estimation still requires fleshing out - it is presumed that
the upgrade feature will be implemented last and the specification will
be refined until then.



\section{Feature: Outputs}

\subsection{Description}

The simulation requires output of runtime traces (for troubleshooting or
self tests) and some collected and aggregated data (for user evaluation of
simulator scenario results).


\subsection{Stimulus/Response Sequences}

Some of the output events are triggered during the simulation, but the
main output


\subsection{Functional Requirements}


\subsubsection{BLOCKSIM-OUTPUTS-REQ-1 (pre simulation outputs)}

Before the simulation is started, the following statistics shall be output
to the user:

\begin{enumerate}
   \item Total budget of all spam-producing traffic generators
\end{enumerate}


\subsubsection{BLOCKSIM-OUTPUTS-REQ-2 (post simulation outputs)}

After the simulation has concluded, the following statistics shall be output
to the user:

\begin{enumerate}
   \item Remaining mining nodes
   \item Remaining service nodes
   \item Aggregate delay for `legit' transactions: number of transactions x number of blocks before inclusion
   \item Aggregate time where $>$ 50 percent of mempools are desynchronized
   \item Cost incurred by spammer
   \item Cost incurred by Poison blockers: if block is orphaned, assume the entire block reward is razed for the attacker
   \item Maximum numbers of blocks that are re-organized during the simulation
   \item Total upgrade cost
   \item Total revenue
\end{enumerate}


\chapter{Other Non-functional Requirements}

\section{Performance Requirements}
TBD

Should be described in terms of feasible network sizes of interest (up to 2000 nodes?),
realistic mempool capacity distributions, transaction generation rates and simulation
time ranges possible on a reasonably modern multi-core PC with disk and memory specs,
available to COTS hardware.


\section{Safety Requirements}
None.

\section{Security Requirements}
None.

\section{Software Quality Attributes}
It is important the simulator be easily configurable and adaptable.

Python is recommended as it is widely available across platforms and supported
by common development environments.

Most Bitcoin Cash developers would be able to do the Python programming needed
to implement further blocksize algorithms for testing, and even enhancements
to the simulator.

Python provides good libraries and unit testing facilities, and also modules
for a possible GUI that could be constructed to make handling the simulator
easier in future.


\section{Business Rules}
None.


\chapter{Other Requirements}
None / TBD.

\section{Appendix A: Glossary}
%see https://en.wikibooks.org/wiki/LaTeX/Glossary
$<$Define all the terms necessary to properly interpret the SRS, including
acronyms and abbreviations. You may wish to build a separate glossary that spans
multiple projects or the entire organization, and just include terms specific to
a single project in each SRS.$>$

\section{Appendix B: Analysis Models}
$<$Optionally, include any pertinent analysis models, such as data flow
diagrams, class diagrams, state-transition diagrams, or entity-relationship
diagrams.$>$

\section{Appendix C: To Be Determined List}
$<$Collect a numbered list of the TBD (to be determined) references that remain
in the SRS so they can be tracked to closure.$>$

\section{Appendix D: To Do List}
This is reserved for a list of changes that still need to be taken into account in
future iterations of this specification.




\end{document}
