%Software Requirements Specification for Adjustable Blocksize Limit Scenario Simulator

%Based on SRS-Tex template:
%Copyright 2014 Jean-Philippe Eisenbarth
%This program is free software: you can
%redistribute it and/or modify it under the terms of the GNU General Public
%License as published by the Free Software Foundation, either version 3 of the
%License, or (at your option) any later version.
%This program is distributed in the hope that it will be useful,but WITHOUT ANY
%WARRANTY; without even the implied warranty of MERCHANTABILITY or FITNESS FOR A
%PARTICULAR PURPOSE. See the GNU General Public License for more details.
%You should have received a copy of the GNU General Public License along with
%this program.  If not, see <http://www.gnu.org/licenses/>.

%Based on the code of Yiannis Lazarides
%http://tex.stackexchange.com/questions/42602/software-requirements-specification-with-latex
%http://tex.stackexchange.com/users/963/yiannis-lazarides
%Also based on the template of Karl E. Wiegers
%http://www.se.rit.edu/~emad/teaching/slides/srs_template_sep14.pdf
%http://karlwiegers.com
\documentclass{scrreprt}
\usepackage{listings}
\usepackage{underscore}
\usepackage[bookmarks=true]{hyperref}
\usepackage[utf8]{inputenc}
\usepackage[english]{babel}
\hypersetup{
    bookmarks=false,    % show bookmarks bar?
    pdftitle={Software Requirement Specification},    % title
    pdfauthor={Jean-Philippe Eisenbarth},                     % author
    pdfsubject={TeX and LaTeX},                        % subject of the document
    pdfkeywords={TeX, LaTeX, graphics, images}, % list of keywords
    colorlinks=true,       % false: boxed links; true: colored links
    linkcolor=blue,       % color of internal links
    citecolor=black,       % color of links to bibliography
    filecolor=black,        % color of file links
    urlcolor=purple,        % color of external links
    linktoc=page            % only page is linked
}%
\def\myversion{0.1 }
\date{}
%\title
\usepackage{hyperref}
\begin{document}

\begin{flushright}
    \rule{16cm}{5pt}\vskip1cm
    \begin{bfseries}
        \Huge{SOFTWARE REQUIREMENTS\\ SPECIFICATION}\\
        \vspace{1.9cm}
        for\\
        \vspace{1.9cm}
        Adjustable Blocksize Limit Scenario Simulator\\
        \vspace{1.9cm}
        \LARGE{Version \myversion (draft)}\\
        \vspace{1.9cm}
        Prepared by Bitcoin Cash developers\\
        \vspace{1.9cm}
        \today\\
    \end{bfseries}
\end{flushright}

\tableofcontents


\chapter*{Revision History}

\begin{center}
    \begin{tabular}{|c|c|c|c|}
        \hline
	    Name & Date & Reason For Changes & Version\\
        \hline
	    imaginary_username & 2019-05-12 & initial requirements & 0.0\\
        \hline
	    freetrader & 2019-05-13 & initial SRS draft & 0.1\\
        \hline
    \end{tabular}
\end{center}

\chapter{Introduction}


\section{Purpose}
This document describes all the requirements for a blocksize simulator
(hereafter shortened to `simulator').

These requirements do not yet map to an actual implementation release
as the simulator has not yet been developed.

This SRS describes the entire simulator system.


\section{Document Conventions}
To be completed (TBC)


\section{Intended Audience and Reading Suggestions}
This document is intended for software developers and users of the simulator.

Software developers are encouraged to develop their own implementations based
on these requirements.

Users may find the requirements useful to understand what the simulator is
intended to do.



\section{Project Scope}
The purpose of the simulator is to test a narrow class of adjustable
blocksize schemes for Bitcoin Cash, where consensus blocksize limits are
adjusted with only constants and actual blocksizes as inputs.

Using a simplified model of a Bitcoin Cash network comprised of resource-constrained
but upgradeable mining and service nodes, and including sources of `legitimate'
and `spam' transactions as well as `poison block' providers, the simulator is
intended to provide information on the following observables:

\begin{itemize}
\item response characteristics of various dynamic blocksize algorithms
\item induced transaction delays in face of full blocks (whether from legit or spam traffic)
\item memory pool (mempool) desynchronization
\item network splits
\item node downtime (both service and mining)
\item upgrade feasibility
\item cost needed for a malicious actor to incur damage to the network
\end{itemize}


\section{References}
TBC


\chapter{Overall Description}


\section{Product Perspective}
The simulator is intended to inform the selection of a dynamic blocksize algorithm
for Bitcoin Cash.

It is intended to be standalone, self-contained, easily adaptable and
re-usable in future.

Test scenarios and algorithms subjected to simulation may be maintained
outside of the simulator project.


\section{Product Functions}
Basic description of the main functions:
\begin{itemize}
\item Interpret command line arguments provided by user to start a simulation run
\item Read scenario inputs from user-provided text files
\item Optionally, generate a block history as initial condition for the simulation
\item Run the network simulation for n blocks, optionally outputting traces to console or log file
\item Stop and output simulation results (to console or disk)
\end{itemize}



\section{User Classes and Characteristics}
The simulator is intended to be used by anyone with an interest in Bitcoin Cash
dynamic blocksize algorithms.

This includes protocol developers, miners, service providers and hobbyists.

Operation of the simulator will require only editing of text files and issuance of
commands in a shell (i.e. at least initially it will be an interpreted command line
program).

Interpretation of the simulator output will likely require domain expert knowledge.


\section{Operating Environment}
The simulator is intended to run on standard commodity personal computer hardware
equipped a suitable interpreter and any required libraries.


\section{Design and Implementation Constraints}
TBC

\section{User Documentation}
TBC


\section{Assumptions and Dependencies}
TBC


\chapter{External Interface Requirements}

\section{User Interfaces}
Initially, the simulator user interface will be text console driven with input
provided by command line and configuration text files, and output written to
console and/or disk.

\section{Hardware Interfaces}
None.

\section{Software Interfaces}
The software is stand-alone and uses OS facilities for console and file I/O.

No other software interfaces are planned.

\subsection{Scenario Files}
It is determined they will exist, but:

Precise structure: To be decided (TBD)

\subsection{Blocksize algorithms}
To be specified as an interface which has to be implemented for a node.
It will have access to the node's view of the blockchain in order to make
its calculations based on past block sizes.

The same algorithm will be used by all the nodes in a simulation run.

\section{Communications Interfaces}
None.


\chapter{System Features}

This section describes the main features of the simulation and lists the
functional requirements associated with them.


\section{Feature: Network}

\subsection{Priority}

High


\subsection{Description}

The simulation requires modeling of a network layer on which
transactions and blocks are transported.


\subsection{Stimulus/Response Sequences}

Traffic generators create transactions which are relayed to all nodes.

Mining nodes create blocks which are relayed to all other nodes (mining, service and generic).

Propagation time of transactions and blocks is not modeled (at least, initially).

That means blocks and transactions take only the minimum amount of time (1 second) to
reach other entities in the simulation.


\subsection{Functional Requirements}


\subsubsection{BLOCKSIM-NETWORK-REQ-1}

Any transactions added to the network by traffic generators at a specific time step shall be inserted into the receive
queue of all nodes at the next simulation time step.

\subsubsection{BLOCKSIM-NETWORK-REQ-2}

Any block added to the network by a mining node at a specific time step shall be inserted into the receive
queue of all other nodes at the next simulation time step.



\section{Feature: Time}

\subsection{Priority}

High


\subsection{Description}

The simulation requires modeling of time (for example w.r.t. the
processing of blocks by nodes).

Certain attributes of simulation entities such as rates of transactions
created by traffic generators, or processing units required by a node
to validate a transaction or block, may be specified most naturally in terms
of some units per time.

Traffic generators will generate transactions to be inserted at a given time
in the simulation (which depends on each generator's characteristics).

Miners will stochastically generate blocks (provided they have transactions
in their mempool). Once a miner finds a block, it adds that block onto
the network immediately.


\subsection{Stimulus/Response Sequences}

The smallest time unit of the simulator will be fixed.


\subsection{Functional Requirements}


\subsubsection{BLOCKSIM-TIME-REQ-1}

The smallest time unit in the simulation will be one (1) second.

\subsubsection{BLOCKSIM-TIME-REQ-2}

The starting time of the simulation will be referred to as time zero
(t=0), all future events will have an integer simulation time > 0
assigned to them.

Events described in scenarios must be expressed in terms of this
simulation time.

Events which occur before the simulation start (e.g. pre-existing blocks
which as part of initial conditions) shall have negative integer simulation
times assigned to them (e.g. t=-1, t=-43, etc.)


\section{Feature: Scenarios}

\subsection{Priority}

High


\subsection{Description}

A scenario completely defines all the inputs for a simulation run.

It may be specified in one single file or a collection of files (TBD),
but in either case there shall be one single file that acts as the
main starting point to provide the input data used in a simulator run.

Some example scenario files can be provided with the base simulator.

Editing scenario files is outside the scope of this SRS. As they will be pure
text files, it can be left to the user and their favorite text editor.


\subsection{Stimulus/Response Sequences}

When users start the simulator, they will need to specify the (main)
scenario file on the command line.

The simulator processes the scenario file and any potential dependencies.

If there are no errors (such as missing or invalid data), the simulation
will run, otherwise the simulator should exit with appropriate error messages.


\subsection{Functional Requirements}


\subsubsection{BLOCKSIM-SCENARIO-REQ-1}

The simulator shall read in the scenario file provided on the command line and
automatically process any dependencies (further included scenario files).


\subsubsection{BLOCKSIM-SCENARIO-REQ-2}

The simulator shall abort if any required input information is missing or
invalid after processing the scenario file(s).

\subsubsection{BLOCKSIM-SCENARIO-REQ-3}
Any missing/invalid scenario information shall be listed to the user to enable
them to adequately configure the scenario.


\subsubsection{BLOCKSIM-SCENARIO-REQ-4}
Input data items specified by scenario shall be as follows
(the list below should be complete):

\begin{enumerate}
\item Default number of simulation steps (blocks) to run. This may be overridden by a command line parameter.
\item Blocksize algorithm to use in nodes (this also defines the `hard' max blocksize for a node)
\item Definition of the node population
   \begin{enumerate}
   \item Number of mining nodes
   \item Number of service nodes
   \item Number of generic nodes (optional - generic nodes would be non-mining, non-service nodes)
   \item Default node mempool size (in bytes)
   \end{enumerate}
\item Definition of the mining nodes
\item Definition of the service nodes
\item Definition of the traffic generators (these are not nodes, but sources which inject transactions traffic into the simulation)
   \begin{enumerate}
   \item sources of `legit' transactions
   \item sources of `spam' transactions
   \end{enumerate}
\end{enumerate}


\section{Feature: Node class}
\subsection{Priority}

High

\subsection{Description}

The node class provides the base from which all node types in the
system (i.e. mining, service, generic nodes) are derived.


\subsection{Stimulus/Response Sequences}

After the scenario has been read and validated, the simulator will
instantiate node objects of the specified types before starting the
simulation.

Specialized node types may override some of the resource attributes
of the node base class.

Node objects maintain their own mempool and view of the blockchain history,
and receive transactions (from traffic generators) and blocks (from miners)
for processing.

If their resources are exceeded during simulated operation, nodes
become unavailable (`downtime`) and may require simulated maintenance (TBD).

Blocks that have been stuck on a Poison block for too long must be able to
fall back to processing another legitimate block instead.


\subsection{Functional Requirements}


\subsubsection{BLOCKSIM-NODEBASE-REQ-1}

Nodes shall maintain their own mempool containing transactions already seen
but not removed by a block in the node's active blockchain history.

\subsubsection{BLOCKSIM-NODEBASE-REQ-2}

Nodes shall maintain their own history of the blockchain, as a list of
linked blocks which the node has been able to `validate`.

\subsubsection{BLOCKSIM-NODEBASE-REQ-3}

Nodes shall have an overrideable attribute which defines a processing rate
at which they are able to validate blocks.

This rate may be defined in terms of some unit (e.g. some `cycles` or `ops`)
used by blocks to specify their validation cost.

\subsubsection{BLOCKSIM-NODEBASE-REQ-4}

Nodes shall maintain a queue for blocks received from miners but not
yet validated.

NOTE: this may include both reasonably sized blocks, and `poison blocks`
which are blocks whose validation may not be reasonably achieved by a
node within a 10 minute timeframe.

\subsubsection{BLOCKSIM-NODEBASE-REQ-5}

If a node is unable to process a `poison block` within a simulated
time of 10 minutes, it shall fall back to validating other received
non-poison blocks.

(TODO: should a node who hasn't mined a block be able to tell before
processing whether a block is poisonous or not?)


\section{Feature: Mining nodes}

\section{Feature: Service nodes}

\section{Feature: Generic nodes}

\section{Feature: Transactions}

\section{Feature: Blocks}

\section{Feature: Traffic generators}

\section{Feature: Node upgrades}


\chapter{Other Nonfunctional Requirements}

\section{Performance Requirements}
TBD

Should be described in terms of feasible network sizes of interest (up to 2000 nodes?),
realistic mempool capacity distributions, transaction generation rates and simulation
time ranges possible on a reasonably modern multi-core PC with disk and memory specs,
available to COTS hardware.


\section{Safety Requirements}
None.

\section{Security Requirements}
None.

\section{Software Quality Attributes}
It is important the simulator be easily configurable and adaptable.

Python is recommended as it is widely available across platforms and supported
by common development environments.

Most Bitcoin Cash developers would be able to do the Python programming needed
to implement further blocksize algorithms for testing, and even enhancements
to the simulator.

Python provides good libraries and unit testing facilities, and also modules
for a possible GUI that could be constructed to make handling the simulator
easier in future.


\section{Business Rules}
None.


\chapter{Other Requirements}
None / TBD.

\section{Appendix A: Glossary}
%see https://en.wikibooks.org/wiki/LaTeX/Glossary
$<$Define all the terms necessary to properly interpret the SRS, including
acronyms and abbreviations. You may wish to build a separate glossary that spans
multiple projects or the entire organization, and just include terms specific to
a single project in each SRS.$>$

\section{Appendix B: Analysis Models}
$<$Optionally, include any pertinent analysis models, such as data flow
diagrams, class diagrams, state-transition diagrams, or entity-relationship
diagrams.$>$

\section{Appendix C: To Be Determined List}
$<$Collect a numbered list of the TBD (to be determined) references that remain
in the SRS so they can be tracked to closure.$>$

\end{document}
